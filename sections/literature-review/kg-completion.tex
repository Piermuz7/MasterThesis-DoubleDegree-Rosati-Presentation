\begin{tframe}{Literature of Knowledge Graph Completion}
\vspace{-.3cm}
Current real-world KGs are usually incomplete and need an inference engine to predict links and complete the missing facts among entities available in the KG.
\newline
\\Knowledge Graph Completion (KGC) is one of the popular technologies for knowledge supplement. 
\vspace{0.1cm}

\begin{itemize}
    \item \textbf{Link Prediction:} aims to predict missing relationships between entities in a graph;
    
    % \begin{minipage}[t]{.5\linewidth}
    %     \textbf{Benefits:}
    %     \begin{adv}
    %         \item Data augmentation;
    %     \end{adv}
    % \end{minipage}%
    % \hfill%
    % \begin{minipage}[t]{.5\linewidth}
    %     \textbf{Challenges:}
    %     \begin{disadv}
    %         \item Scalability, Data sparsity, Bias, ;
    %     \end{disadv}
    % \end{minipage}
    % \vspace{.1cm}
    \item \textbf{Knowledge Graph Embedding (KGE)}: is a representation of a KG element into a continuous vector space;
    % \begin{minipage}[t]{.5\linewidth}
    %     \textbf{Benefits:}
    %     \begin{adv}
    %         \item GNNs models learn\\ powerful embeddings by\\ using topological structures\\ of KGs;
    %     \end{adv}
    % \end{minipage}%
    % \hfill%
    % \begin{minipage}[t]{.5\linewidth}
    %     \textbf{Challenges:}
    %     \begin{disadv}
    %         \item Treat each triple independently;
    %         %\item fail to cover the complex inherently implicit information in the local neighborhood of a triple;
    %         \item Fail to cover the complex inherently implicit information.
    %     \end{disadv}
    % \end{minipage}
    \item \textbf{GNNs} have been recently applied in KGs to learn powerful embeddings by using topological structures in the KGs.
\end{itemize}
\end{tframe}